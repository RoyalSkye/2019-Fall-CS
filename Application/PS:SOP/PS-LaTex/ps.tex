\documentclass[16pt,letterpaper]{ctexart}
\usepackage[T1]{fontenc}
\usepackage{fancyhdr}
%\usepackage[utf8]{inputenc}
\usepackage[margin=0.8in]{geometry}

\newcommand{\HRule}{\rule{\linewidth}{0.5mm}}
\newcommand{\Hrule}{\rule{\linewidth}{0.3mm}}

\makeatletter% since there's an at-sign (@) in the command name
\renewcommand{\@maketitle}{%
  \parindent=0pt% don't indent paragraphs in the title block
  \centering
  {\Large \bfseries\textsc{\@title}}
  \HRule\par
  \textit{\@author \hfill \@date}
}
\makeatother% resets the meaning of the at-sign (@)

\title{Statement of Purpose}
\author{Jianan Zhou}
\date{12/28/2018}
\linespread{1.2} 

\begin{document}
  \maketitle% prints the title block
  \thispagestyle{empty}
  \vspace{12pt}

I want to pursue a Master’s degree in Electrical and Computer Engineering at Carnegie Mellon University so that I can better understand the field of electrical and computer engineering, and build up my career as a globally-competent researcher. I have prepared myself for the next step in my professional and academic journey through my undergraduate degree by studying challenging coursework, engaging in cutting-edge research, and making dedicated efforts to diversify myself and my experiences through extracurricular activities. The undergraduate curriculum at Northeastern University has provided me with a concrete background in math, software engineering, computer science, and a wide spectrum of wide range of engineering knowledge. I thrive under pressure and studying at one of the top software engineering programs in China has given me the motivation to rank among the top students in the department.

Strong mathematical abilities are what initially got me interested in algorithms. During my freshman year of university, I participated in ACM school team practice, which made me really interested in applying algorithms to solve all kinds of compelling problems. In my junior year, I enrolled in the course Introduction to Artificial Intelligence, which furthered my interest in algorithms. In this class, Professor Hao introduced basic knowledge such as the heuristic algorithm, which I applied to solve the eight puzzle problem. This course experience inspired me to attend a collegiate programming contest and got a prize during the second half of my junior year. My desire to apply my knowledge of mathematics and data structures to solve convolutional challenges motivated me to further my studies in ECE.

Game development in Linux made me narrow down my orientation for my future career. In my junior year, the final project in my course Linux Operating System, which was one of the most impressive projects, is creating a game using what we learnt from the class. Although some students called various libraries to help develop their projects, I developed a multiplayer online game with C. As I encountered problems, I would search in Linux man page and online resources. Although this project was very time consuming, I felt incredible achievement after completing it. Through this valuable experience, I realize that I have more enthusiasm in creating programs rather than applying existing libraries. To be an innovative developer is my dream, and I need to continue my study to achieve this goal.

Doing a research project directed by Professor Guo during the second half of my junior year aroused my interest in the recommendation system, even research. During the research, I read many theses on personalized recommendations using knowledge graphs and the application of Knowledge Graph Embedding in recommendation systems. However, my undergraduate education focuses on fundamentals and important current issues in computer science and computer engineering, which means that there are few opportunities for students to choose their desired areas of specialization to do some research. To have more of this kind of opportunities is another important reason I want to pursue a master’s degree in ECE.


In addition to my academic studies at university, my internship experiences have also played a significant role in my professional development. As one of the team leaders in an intern project, I realized the importance of team cooperation. At the end of my junior year, I became a Java Web intern in Neusoft Corporation. At the beginning of the development, our team was not efficient since I did not assign the work reasonably to any members. I was faced with the question of how to distribute each employee’s work to increase the efficiency of development. Further: What is the best way to maintain team harmony and make communication between them efficient and effective?  These are critical problems that a qualified system architect and team leader should consider in order to maximize the efficiency of project development. Furthermore, during the development of our web project, we used many tools such as redis, nginx and others to improve server performance and customer experience. I really interested in the idea of the tools’ implementation and design. This also empowers me to pursue continuous learning in a graduate program. 

As a senior, I have realized that while I have learned great skills and developed strong relationships and a basic understanding of what I want to do in the future, what I learned from my undergraduate classes is too general to be utilized in real research in any specialized field. Graduate school will provide me with intensive preparation in the concepts and techniques related to the design, programming, and application of computing systems and provide me the opportunity to further my knowledge and understanding of novel technologies in the field. Now, with my interests in algorithms, software systems and data science, enthusiasm for learning and creating, as well as fundamental knowledge I have learned during my undergraduate program, I know it is the right time to pursue a Master’s degree in ECE to access advanced technology and prepare for my future career. 

In my opinion, the master’s program at Carnegie Mellon University would help me gain exposure to multi-dimensional training through a practical and in depth implementation of projects and a theoretical understanding of electrical and computer engineering principles. I am captivated by its distinguished faculty and interdisciplinary curriculum such as 11785 Introduction to Deep Learning and 15619 cloud computing. With the comprehensive degree plan and various resources provided by the program, we may have a thorough background in the fundamentals of electrical and computer engineering and software engineering, as well as the opportunity for in-depth specialization in some particular aspect of these fields, so that we may either obtain productive employment or pursue advanced degrees in the future.

I think I have the ability to further my studies in the master’s program at Carnegie Mellon University. As you can see, my increasing study trend from my academic transcript and the work, project and research experiences I have undertaken drive me to maintain my enthusiasm of studying and pursuing my dream at CMU. Furthermore, with my learning ability, my specific experiences, and my culture background, I am confident that I will diversify this program and give vigor and energy to CMU. “Follow my heart, keep hungry and keep stupid”, a quote by Steve Job, has always been a driving force to remind me to follow my interests and keep my urge of pursuing knowledge to further my study at a graduate level. I have faith that I am the ideal candidate for the electrical and computer engineering master program at CMU, where I can fulfill my dream.

\end{document}